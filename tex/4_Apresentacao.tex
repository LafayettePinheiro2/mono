
Este capítulo busca mostrar uma visão sobre o curso de Licenciatura em Computação, seu histórico no Brasil, seus desafios, potencialidades e propostas á evolução do curso.

%%%%%%%%%%%%%%%%%%%%%%%%%%%%%%%%%%%%%%%%%%%%%%%%%%%%%%%%%%%%%%%%%%%%%%%%%%%%%%%%
%%%%%%%%%%%%%%%%%%%%%%%%%%%%%%%%%%%%%%%%%%%%%%%%%%%%%%%%%%%%%%%%%%%%%%%%%%%%%%%%
%%%%%%%%%%%%%%%%%%%%%%%%%%%%%%%%%%%%%%%%%%%%%%%%%%%%%%%%%%%%%%%%%%%%%%%%%%%%%%%%
\section{O curso / histórico}%
 
O curso de Licenciatura da Computação surgiu no Brasil em 1997, na Universidade de Brasília (UnB). Trata-se de um curso interdisciplinar, que integra em sua grade curricular bases pedagógica, científica e tecnológica para a formação de profissionais que lidem com o ensino de informática e a capacitação técnica, além do tratamento da informática na educação. Para Matos (2013) o curso possui um grande desafio curricular, o de promover a interdisciplinaridade com base no "movimento dialético entre os saberes pedagógicos, científicos e tecnológicos no processo de formação relativo ao perfil profissional almejado".

Trata-se de um curso relativamente novo, completando 20 anos de implantação no Brasil em 2017, e a Universidade de Brasília mostra-se como pioneira sendo a primeira a implantá-lo a nível nacional. É um curso que compreende o desafio do novo, do que ainda não foi visto ou feito antes. E para isso requer uma base teórica compatível, uma base de prática profissional e a coragem e vontade de inovar.

A atuação dos licenciados em computação não deve ser o ensino puro de conceitos técnicos de informática, mas sim as habilidades a serem trabalhadas para que os alunos possam realizar atividades didático-pedagógicas através das potencialidades oferecidas pela tecnologia da informação, exercendo um uso crítico-reflexivo das TIC. Castro (2013) vê os licenciados em computação como agentes integradores das tecnologias da computação, no processo de ensino-aprendizagem, sendo capazes de compreender o fenômeno educativo na sua diversidade e na sua complexidade, contextualizando-o social e tecnologicamente no seu tempo e no seu espaço.
 
Começa-se a compreender o papel estratégico na sociedade do professor de computação enquanto capacitador técnico e integrador do binômio computação-educação, podendo colaborar com o desenvolvimento da computação, da educação além de ter um importante papel de intervenção social. Matos (2013) oportunamente traz questionamentos acerca da formação deste profissional: quais seriam os saberes específicos do profissional docente em informática presente nas escolas de educação básica, nas instituições de nível superior ou nas corporações? Seria o conjunto de saberes de todas as disciplinas escolares? Seria o saber puro (técnico-científico) da Computação? Ou seria um saber específico e interdisciplinar?
 
Segundo o Currículo de Referência formulado pela SBC (FAZER REFERÊNCIA DA SBC), o curso de licenciatura deverá focalizar a formação especializada e multidisciplinar, de modo que o seu egresso possa atuar na educação básica nas escolas, para as séries finais do ensino fundamental e para o ensino de nível médio, e a educação profissional, para as demandas produtivas do trabalho de formação geral e especializada. Ambos os campos de atuação do licenciado podem ter a computação como o corpo de conhecimentos.

O curso foi pensado a partir da análise das necessidades sociais, políticas, econômicas nacionais, buscando a formação tecnológica, complementar e humanística, integrando a pesquisa e prática pedagógica, bem como estágios supervisionados. A importância deste profissional é primordial na atualidade em que a computação está integrada à maioria dos segmentos da sociedade. 

Matos (2013) contribui afirmando que a formação ampla e interdisciplinar na LC propõe saberes pedagógicos teóricos e empíricos para que o professor de computação, enquanto profissional, possa desenvolver competências específicas para sua atuação na escola, contribuindo eficazmente para o aprendizado técnico e crítico dos seus alunos, bem como para o uso da informática na educação como ferramenta de promoção da cidadania. Ainda segundo o autor, as técnicas são de extrema importância, porém deve-se trabalhar o lado humanístico também, pelo fato que esta ação trás uma maior proximidade entre aluno-professor, e contribui para o processo de ensino-aprendizagem. Este é um curso diferenciado no sentido de ser um curso da área da tecnologia e das ciências exatas, que costuma ser mais focado na parte técnica, e envolve educação, integrando o lado humanístico ao curso. E no contexto da interdisciplinaridade, é possível aproveitar-se dos avanços teórico-metodológicos de todas as áreas envolvidas, colaborando com o crescimento destas em conjunto.


			ADICIONAR IMAGEM 
			Tríade de saberes da Licenciatura em Computação
			Fonte: MATOS 2013
 
As diretrizes curriculares dos cursos de computação e informática (Brasil 1999) incentivam a formação voltada para desenvolvimento de competências técnicas, científicas e pedagógicas integradas, caracterizando uma formação multidisciplinar. Tal inclui a concepção tradicional do perfil do licenciado e sua formação docente, bem como aspectos tecnológicos (modelagem de ferramentas computacionais de software e hardware para uso educacional, aquisição, instalação e gestão de recursos de TI aplicados à educação).

Além disso, as mesmas diretrizes evidenciam a relação da qualificação profissional do egresso no curso de licenciatura em computação com a visão crítica, a reflexão e a busca de novas formas de ensino, adaptando-se criativamente em um tempo de inovações constantes. Segundo estas diretrizes, o profissional deve estar apto a desenvolver softwares visando a qualidade do ensino e da aprendizagem. Em sua atividade profissional, esse profissional deve buscar inovação e ter a pesquisa como ferramenta de trabalho do cotidiano, de modo a compreender as dificuldades discentes, trabalhar conteúdos éticos, culturais e sociais que vão edificar o caráter do aluno. 

Junior (et al, 2014) chama a atenção para a necessidade de compreender que as mudanças tecnológicas geraram diversos avanços no processo produtivo como um tudo, demandando um curso que venha trazer desenvolvimento tecnológico para as regiões. Atualmente é difícil imaginar uma sociedade desenvolvida ou que busca o desenvolvimento sem o conhecimento e uso das tecnologias da informação. A área da computação é estratégica por estar integrada em diversas atividades humanas, "das artes às tecnologias". Ainda segundo o autor, os cursos de formação superior na área de computação têm como objetivo principal a formação de recursos humanos qualificados para apoiar o desenvolvimento tecnológico destas áreas com vistas a atender as necessidades da sociedade. Neste contexto, as necessidades da sociedade podem ser compreendidas como sendo aquelas atividades, sistemáticas ou não, que podem ser desenvolvidas com o auxílio de um sistema computacional. No Brasil, diferentes políticas de formação já foram concretizadas neste campo de trabalho, integrando tecnologias e processos cognitivos.

Assim, cabe destacar a importância da licenciatura em computação, que abrange desde a formação especializada de educadores na área computacional até a formação de profissionais, que por opção quiserem atuar no mercado de trabalho com atividades específicas da computação, por mais que este último ponto não consista no objetivo principal do curso. JUNIOR (et al, 2014) destaca a necessidade deste curso em formar profissionais de atitudes, reflexão, criticidade e aptos a fazer escolhas, por estar integrado em um meio de constante mudança e inovação. 


Conforme Cambraia e Scaico (2013), o licenciado em Computação é um professor da educação básica que tem como missão pensar o uso efetivo das tecnologias na escola, para uma apropriação dos benefícios que podem surgir com as redes de informação, as tecnologias sociais e o conteúdo digital de caráter instrucional, e também de ser um educador da Ciência da Computação enquanto uma ciência. A importância deste profissional na sociedade surge, sobretudo, por acreditar no potencial que a Computação possui para contribuir com o desenvolvimento de cidadãos mais capacitados e com um modelo de escola que compreenda o novo papel que o uso da tecnologia pode assumir frente a um mundo conectado e em constante mudança.


Além da formação específica relacionada às diferentes etapas da Educação Básica, este curso deve buscar a inserção no debate contemporâneo mais amplo, envolvendo questões culturais, sociais, econômicas e o conhecimento sobre o desenvolvimento humano e a própria docência, já que esses futuros professores atuarão em situações ímpares, indeterminadas e incertas, o que lhes exige criatividade e iniciativa para tomadas de decisão, que dificilmente encontrarão modelos já prontos que a formação inicial lhes proporciona. Todo o curso é incentivado à construção de projetos inovadores e próprios (CRUZ 2013).


Prietch (2009) aponta a necessidade do curso em manter seus Projetos Políticos Pedagógicos atualizados, haja vista do meio de constante mudanças em que se encontra, tornando o processo de inovação ainda mais complexo. 


As Diretrizes curriculares nacionais em Ciência da Computação (Brasil 1999), na parte específica da Licenciatura em Computação, propõem que a formação docente seja pensada sob o ponto de vista da interdisciplinaridade das esferas que a compõe: acadêmica, científica, tecnológica e corporativa. Assim, os conhecimentos didático-pedagógico não deveriam estar desvinculados do conhecimento específico, seja ele científico ou tecnológico. Enquanto os professores de um curso de Licenciatura ensinam um conhecimento específico, uma rica experiência tanto para o professor quanto para os alunos (futuros professores) seria pensar em como está sendo efetuado o processo de ensino-aprendizagem deste tópico específico, se o método está sendo efetivo ou não, possíveis alternativas, etc. Aproveitar no processo a variedade de conhecimentos que os discentes já trazem consigo, e pensar reflexivamente para quando forem eles os professores. As relações entre teoria e prática que permitem ao egresso adaptar-se, com visão crítica, às novas situações de sua área de formação.


Para isso a formação acadêmica deve incluir qualificação pedagógica satisfatória, formação técnica e científica dialeticamente articuladas visando desenvolver competências específicas para a atuação na escola, contribuindo eficazmente para o aprendizado técnico-crítico dos seus alunos e para a informática na educação como ferramenta de promoção da cidadania.


São importantes características do perfil do profissional do professor de computação o desenvolvimento da iniciativa própria e a capacidade de enfrentar novas situações, obtidas por meio de um currículo abrangente, da prática em situações reais e da relação ensino-aprendizagem participativa. Assim, o profissional deveria ter autonomia na busca de informações e uma base conceitual que lhe permitisse acompanhar a rápida evolução do conhecimento nessa área do saber (BEZERRA 20XX).


Mesmo com a necessidade da sociedade de tal profissional, Matos (2013) aponta oportunamente a necessidade de uma reflexão mais profunda sobre esse profissional que tem concluído o curso superior com pouco reconhecimento pela área de Computação, bem pela Educação. Não existem disciplinas específicas nos currículos da educação básica para atividade profissional regular dos egressos dos cursos de LC. Além de desafios nas questões epistemológicas, há também desafios em questões de poder, trabalho e reconhecimento profissional / social.	
 

%%%%%%%%%%%%%%%%%%%%%%%%%%%%%%%%%%%%%%%%%%%%%%%%%%%%%%%%%%%%%%%%%%%%%%%%%%%%%%%%
%%%%%%%%%%%%%%%%%%%%%%%%%%%%%%%%%%%%%%%%%%%%%%%%%%%%%%%%%%%%%%%%%%%%%%%%%%%%%%%%
%%%%%%%%%%%%%%%%%%%%%%%%%%%%%%%%%%%%%%%%%%%%%%%%%%%%%%%%%%%%%%%%%%%%%%%%%%%%%%%%
\section{Propósitos e potencialidades do curso}%
 
Muito se divulga sobre os cursos de Licenciatura em Computação terem como objetivo primordial a formação de profissionais para atuar no ensino de computação e informática no nível médio. Tal objetivo é verdadeiro, mas as formas de atuação do licenciado não podem se limitar apenas a isso, há uma necessidade de compreender o crescente campo de atuação deste profissional e sua real perspectiva de atuação frente a uma sociedade majoritariamente informatizada.	

	Na página do departamento de Ciência da Computação da Universidade de Brasília, descreve-se que 
    
(((((((citaçaõ direta letra menor à direita))))))))
\begin{quotation}
a Licenciatura em Computação, que é realizada no período noturno e tem duração de 4,5 anos, tem por objetivo formar educadores para o ensino de computação e informática das escolas da rede pública e particular dos ensinos fundamental e médio; da educação profissional, para a qualificação para o trabalho; das empresas, onde a computação é a base da formação para treinamento e educação coorporativa. O curso visa atender à necessidade imediata de informatização e absorção dos avanços dessa área nos diversos setores da sociedade. Para isso, o curso conta com disciplinas que integram as áreas de computação e educação.
\end{quotation}
 
	Cabral et al. (2008, p. 17) indicam que também são propósitos do curso: 
    \begin{enumerate}
	\item formar recursos humanos para projetar sistemas de softwares para educação a distância; 
	\item formar recursos humanos para projetar softwares, plataformas e objetos educacionais (ferramentas de Software Educacional; informática na escola); e 
	\item formar educadores para o ensino de Computação em instituições que introduzirem a Computação em seus currículos, como matéria de formação. 
	QUIM (2013) adiciona:
	\item colaborar com os outros professores na atuação com as TIC (seja no auxílio à qualificação destes professores ou na integração das TIC no ensino de suas disciplinas);
	\item Coordenar os laboratórios de informática nas escolas; e
	\item Realizar pesquisa acadêmica em informática na Educação.
    \end{enumerate}
 
Castro (2013a) acrescenta a possibilidade de atuação em equipes multidisciplinares para a  "transposição pedagógica de conteúdos disciplinares para tecnologias e metodologias educacionais", visando a transformação do processo de ensino-aprendizagem, bem como o bom uso dos laboratórios de informática das escolas públicas e particulares do país, já que vários destes laboratórios estão fechados ou subutilizados por falta de profissionais capacitados para possibilitar o uso integrado e efetivo. Os licenciados em computação, atuando em equipes multidisciplinares, podem colaborar com a capacitação de outros professores no uso da tecnologia em sala de aula, e colaborar com a inserção tecnológica nas práticas educacionais. 


Guimarães, Sena e Campos (2013) constatam importância do papel do licenciado em computação atuando em projetos interdisciplinares com o uso das tecnologias nos ambientes escolares, revelando em suas pesquisas que a atuação de tal profissional pode colaborar com avanços educacionais significativos em processos interdisciplinares mediados por tecnologias. Os licenciados colaboram ainda com a capacitação dos outros professores para o trato com tecnologias.


A integração dos profissionais de LC em equipes multidisciplinares para trazer novas técnicas e métodos de ensino pode contribuir com o um forma de educação progressista, que, segundo Freire (2011), assume que ensinar não é transferir conhecimento, mas sim criar as possibilidades para a sua produção e construção. Os educadores e educandos devem ser instigadores, inquietos, curiosos, humildes e persistentes ao invés de apenas receber as informações passivamente. Os professores devem estar abertos a indagações, à curiosidade, às perguntas dos alunos, a suas inibições e a lidar com seres críticos e inquiridores.


Conforme Alves e Zambalde (2002), os educadores necessitam estar preparados para uma nova fase da educação na qual é exigida a capacidade de interpretar, compreender, assimilar e processar um número cada vez maior de informações. Os professores precisam estar atentos para não retrocederem a detentores do conhecimento, mas sim tornar-se cada vez mais lançadores de desafios e participantes ativos do processo educacional.


Assim, a integração da tecnologia em sala de aula promove a busca por outras abordagens pedagógicas, que vão além do tradicional, e que buscam o interesse dos alunos, o desafio pelo novo e pela transformação. A compreensão de uma nova visão educacional com a tecnologia em sala de aula é fundamental para inovar e evoluir o processo educacional, acompanhando a rápida modernização social-tecnológica que vem ocorrendo.


	Prietch (2009) ressalta que a carência de profissionais de educação em computação priva as escolas de ensinar disciplinas desta área e também os cursos profissionalizantes. Também influencia na introdução de fundamentos dessa ciência nos currículos regulares da educação fundamental e média. Com mais profissionais de educação em computação seria mais viável a execução de projetos interdisciplinares e transversais, da informatização escolar, dos projetos de softwares educacionais e objetos de aprendizagem de qualidade, dentre muitas outras potencialidades.
    
    
	Além disso, estes profissionais tendem a colaborar fortemente na disseminação da Informática na sociedade brasileira, exercendo da docência e consultorias em escolas públicas e particulares, e a atuação na criação, desenvolvimento e implantação de softwares educativos e objetos educacionais (BEZERRA 20XX).
    
    
	Portanto, é notável que as formas de atuação dos profissionais licenciados em computação são muitas, desde a atuação em salas de aula, à atuação em diversos recursos que integram computação e educação. Novas formas de atuação podem ser criadas, da mesma forma em que a tecnologia evolui aceleradamente. É um mercado que já possui muita demanda e tende a ser ainda mais necessário na sociedade com o passar dos anos e a intensificação da informatização nos diversos segmentos da sociedade. Tal profissão tende a contribuir com o desenvolvimento da computação, da educação e da qualificação da sociedade. Afinal de contas a Tecnologia da Informação é uma profissão do futuro, tem diversas áreas que podem ser focadas e especializadas, surgindo a cada dia mais demandas nesta área. A educação nesta área sempre será importante, permeando muitas atividades da sociedade. 
 
%%%%%%%%%%%%%%%%%%%%%%%%%%%%%%%%%%%%%%%%%%%%%%%%%%%%%%%%%%%%%%%%%%%%%%%%%%%%%%%%
%%%%%%%%%%%%%%%%%%%%%%%%%%%%%%%%%%%%%%%%%%%%%%%%%%%%%%%%%%%%%%%%%%%%%%%%%%%%%%%%
%%%%%%%%%%%%%%%%%%%%%%%%%%%%%%%%%%%%%%%%%%%%%%%%%%%%%%%%%%%%%%%%%%%%%%%%%%%%%%%%
\section{Desafios}%
 
	A licenciatura em Computação enfrenta uma série de desafios desde o momento de sua concepção até hoje em virtude do caráter inovador do curso. Castro (et al, 2013a) afirma que, por ser um curso muito recente e aliar em sua identidade pedagógica duas áreas bem distintas, a LC possui alguns obstáculos, tais como:
    \begin{enumerate}
	\item ausência da disciplina de computação na educação básica e consequente restrição de criação de concursos específicos para profissionais licenciados em Computação: "quais são os papéis de uma LC no contexto educacional do Brasil? por que formar um licenciado em Computação se não há disciplina obrigatória de Computação no ensino formal?" Com a forma com que a informática está integrada na sociedade, o licenciado em computação pode atuar em uma sala de aula ou em diversas outras formas de atuação (construção de softwares educativos e objetos de aprendizagem, na atuação em equipes multidisciplinares, em ambientes de educação formal e não-formal, etc). As possibilidades de atuação são tão promissoras quanto o próprio desenvolvimento tecnológico, e necessita de criatividade e pesquisas na área. Isto não significa que a Computação deva ou não fazer parte de um currículo oficial de educação básica, mas que os seus conhecimentos já fazem parte, imprescindivelmente, da formação intelectual e cidadã, sendo fundamentais para a vida social contemporânea.
	\item falta de reconhecimento da identidade do curso: menor visibilidade e status do curso em relação às outras denominações (ciência / engenharia). O mercado de trabalho,  quer na esfera pública ou privada, em geral desconhece o perfil de um egresso em Licenciatura em Computação, os objetivos do curso, ou mesmo a sua própria existência podendo-se chegar a um cenário onde o egresso não encontra espaço adequado de atuação. O desconhecimento não se limita aos empregadores, como também se estende aos vestibulandos, que não raro desconhecem por completo ou têm concepções errôneas a respeito desse curso, isso quando sabem de sua existência.
	\item desvalorização geral da carreira docente no país: no Brasil a carreira de professor é realmente desvalorizada, enquanto o mercado de trabalho para profissionais de TI é promissor. Torna-se mais atraente para os estudantes trabalhar na iniciativa privada ou pública em funções puramente tecnológicas, do que buscar atuar no viés educacional. A inexistência de políticas públicas de absorção do egresso em escolas municipais, estaduais ou mesmo particulares para a integração ao mercado de trabalho, bem como a falta de políticas de planos de carreira tornam a profissão docente menos atrativa.
	\item dificuldade na integração entre as competências das áreas tecnológicas e pedagógicas: como conciliar os conhecimentos tecnológicos com os saberes pedagógicos? Estes tão estáveis e de difíceis mudanças enquanto aqueles com diversas alterações todos os dias. Como superar preconceitos e  consolidar a formação docente frente a tantas variáveis? Quem reconhece a real importância da integração educação-computação? A educação, a computação ou nenhum dos dois? A matriz curricular busca um trabalho multidisciplinar ou interdisciplinar? O ensino desta ciência na Educação Básica deve ser uma tarefa de professores qualificados para tal?
	\item pouca oferta em Instituições de Ensino Superior (IES): por não ser um curso muito procurado pelos alunos, vem diminuindo o número de instituições de ensino superior ofertando a LC ultimamente. Poucas Universidades Federais a ofertam, e o principal papel na disseminação da LC está a cargo dos Institutos Federais (IF), que estão ampliando sua visibilidade. Poucas IES particulares ofertam-na, revelando uma visão mais mercadológica do Ensino Superior do que uma fonte de transformação social.
	Podemos adicionar ainda mais alguns desafios a estes já citados:
	\item baixo reconhecimento das reais potencialidades de atuação dos profissionais desta área, bem como a força de intervenção e desenvolvimento social e tecnológico que estes profissionais podem oferecer.
	\item falta de suportes para as práticas na área: a falta de computadores, livros, internet, softwares, aplicativos com licenças e outros atrapalha a integração e o desenvolvimento da computação. A estrutura física básica, os equipamentos e a capacitação tecnológica dos professores é fundamental para a Licenciatura em Computação ocorrer eficazmente.
	\item falta de pesquisa acadêmica na área de educação em computação: a pesquisa no binômio educação-computação ainda não atende o tamanho que as demandas nessas áreas são. A área de pesquisa acadêmica ainda está segmentada, fazendo-se necessária a integração. A CAPES e o CNPq, que são os principais fomentadores da pesquisa no Brasil, não reconhecem como área de projetos a serem recebidos a educação em computação. Tais pesquisas podem colaborar fortemente com o desenvolvimento de ambas as áreas, a partir de uma perspectiva conjunta.
	\end{enumerate}
    
    Castro (et al, 2013a) ainda aponta mais desafios, estes enfrentados por toda a educação superior brasileira, como: "profissionalização precoce; matriz profissionalizante; idade média alta dos estudantes do ensino superior; trabalho durante a graduação; baixo volume de estudo discente; e significativa presença de ensino noturno".
	
    
    Frente a tantos desafios, mister se faz aprender a enxergar os problemas e as dificuldades como oportunidades. Á medida em que há desafios para a licenciatura em Computação, o rápido desenvolvimento tecnológico atual e a necessidade de reforma educacional abrem espaço para diversas intervenções nestas áreas, que necessitam de pessoas voltadas para a inovação, a criatividade e a visão do desenvolvimento do país sob uma perspectiva presente / futuro.
	
    
    A computação ainda é vista hoje como uma mera ferramenta, de caráter utilitário. Este cenário está em fase de transição pela força com que a computação está se integrando em todas as áreas da sociedade. Os cursos de Licenciatura em Computação são cada vez mais necessários para a
formação de profissionais qualificados para atuar no campo de Informática na Educação, especialmente em se considerando os desafios do ensino de Computação para a Educação Básica nesta nova realidade imposta pela importância da área, cada vez mais integrada ao cotidiano da sociedade. 

[[[[[[3.** Evasão --> colocada ainda como um desafio]]]]]
 
	Priecht e Pazeto (2009) em seus trabalhos apontam um alto índice de evasão nos cursos ligados à computação, e na licenciatura em Computação a situação não é diferente. Florencio em sua pesquisa apresenta que o curso de LC na UnB apresenta 73\% de evasão de 1997 a 2003*. O autor aponta causas pedagógicas como uma das principais causas para este fenômeno. Santos aborda este problema afirmando que "a carga de conceitos abstratos nos primeiros anos dos cursos da área é significativa e pode ser decisiva para a motivação dos estudantes". Ainda mais no curso de LC que não possui uma identidade bem consolidada socialmente, os alunos entram no curso com uma visão errônea do que este realmente é.
	Cambraia e Scaico (20XX) corroboram que a desinformação (dos alunos de ensino médio/fundamental), a existência de diversos estereótipos e a falha da escola em ajudá-los a entender as possibilidades de carreira na área de computação, são fatos que são danosos para a formação do estudante e que, constatadamente, causam consequências negativas quando estes estudantes entram em cursos superiores, a exemplo dos altos níveis de evasão e reprovação dos ingressantes relatados em pesquisas. Muitos estudantes do ensino básico nunca tiveram um real esclarecimento sobre o que é realmente a Computação e suas possibilidades de atuação profissional. Quando esta temática é direcionada à Licenciatura em Computação, o desconhecimento é ainda maior.
 
%%%%%%%%%%%%%%%%%%%%%%%%%%%%%%%%%%%%%%%%%%%%%%%%%%%%%%%%%%%%%%%%%%%%%%%%%%%%%%%%
%%%%%%%%%%%%%%%%%%%%%%%%%%%%%%%%%%%%%%%%%%%%%%%%%%%%%%%%%%%%%%%%%%%%%%%%%%%%%%%%
%%%%%%%%%%%%%%%%%%%%%%%%%%%%%%%%%%%%%%%%%%%%%%%%%%%%%%%%%%%%%%%%%%%%%%%%%%%%%%%%
\section{Necessidades da LC}%
 
	A Licenciatura da Computação mostra diversas formas de colaboração com a sociedade atual, bem como uma série de desafios a serem enfrentados. Assim, o curso possui algumas necessidades para que seja de fato fomentada a sua importância e a sua melhora. Cruz (2013) atenta para a necessidade de  formação de professores constantemente atualizados nos conhecimentos curriculares, pedagógicos e às novas tendências educacionais. Em meio a tantas mudanças sociais, tecnológicas e econômicas, a educação não pode estagnar suas metodologias, mas sim acompanhar o desenvolvimento da sociedade. Os alunos das novas gerações, muitas vezes "nativos digitais", são normalmente atraídos pela tecnologia, reforçando no professor uma conduta de adaptação constante. Através da formação continuada o professor deve adquirir a habilidade de conjugar o tradicional e o inovador, fazendo releituras de suas aprendizagens iniciais sobre as teorias e as práticas da Educação.
	
    
    Matos e Silva (20XX), ainda sobre a formação do professor, ressalta o fato de ao licenciado em computação tem-se concebido uma promessa de formação ampla. Os autores trazem questionamentos pertinentes acerca desta formação: 
 
Isto [uma formação ampla] seria de fato possível? Esta formação atende às atuais necessidades profissionais? Mediante sua formação curricular, ela é suficiente para poder ingressar no mercado de trabalho? Estas questões abrem terreno fértil para investigações futuras acerca do efetivo conhecimento que tem sido construído ou que pode ser construído no âmbito de cursos de Licenciatura em Computação. 
 
	Existem diversos campos de atuação para este profissional no cenário educacional e computacional. Tal profissão, exercida com eficiência, pode representar diversos avanços para o sistema educacional do País. Sena (2014) defende a necessidade de implantação de políticas públicas para estruturar a carreira do licenciado em computação para que não haja a redução dos cursos de LC por falta de reconhecimento da profissionalização docente na área de computação/informática. É importante que estes profissionais sejam reconhecidos por sua importância na educação brasileira, uma vez que as escolas e a sociedade como um todo já absorveram o uso das tecnologias em seus ambientes.
	
    
    Mister se faz o reconhecimento e a valorização do curso de LC e da profissão do licenciado pelo  governo brasileiro e pela comunidade acadêmica. Para que os estudos sobre a LC tenham um real progresso, é necessária a sua inclusão regular nas estratégias de posicionamento e desenvolvimento da computação no país, contribuindo para o processo educacional nacional, para a valorização do curso e para a atuação profissional do licenciado. A inclusão de políticas públicas de inserção no mercado de trabalho para os egressos, assim como a regulamentação dos cargos e de concursos públicos específicos podem alavancar a profissão. O fomento à formação continuada na área pode fornecer mais pesquisas para a evolução da LC no país, visando a obtenção de estudos aprofundados na Computação-Educação.
	
    
    Matos e Silva (20XX) ainda atentam para a necessidade de pesquisa científica na área, como uma fator fundamental para o desenvolvimento do binômio educação-computação. Para os autores, é fundamental conceber a educação em Computação enquanto area de pesquisa e em diálogo constante e crítico com as ciências da educação. Prietch e Pazeto (2009) também defendem em suas pesquisas a importância das  atividades relativas à extensão e à pesquisa vinculadas ao ensino. No curso há o ensino de disciplinas específicas que objetivam estimular a pesquisa científica, o aprofundamento de conteúdos, e a investigação sistemática de problemas no intuito de obter soluções, bem como a geração e construção do conhecimento. Assim, seria criado um ciclo capaz de alavancar novos conhecimentos e soluções ao tempo em que, a comunidade, as Universidades e a sociedade como um todo serão beneficiadas.
	
    
    Sena e Guimarães (20XX) apontam para a urgência da criação de políticas de formação inicial e continuada de professores, vinculando cursos de licenciaturas a professores em atividade, para colaborarem para uma educação onde a pesquisa e "o aprender a aprender" prevaleçam. Professores que pesquisam e refletem acerca da sua prática tornam os alunos mais ativos e curiosos para a produção do conhecimento. Desta forma, há a necessidade de os alunos da licenciatura interagirem com professores já em atividade, ainda durante a sua formação docente, para poder vivenciar as dificuldades e melhores formas de tratar determinadas situações, compreendendo melhor o contexto em que estarão inseridos profissionalmente. 
	
    
    A construção de novos conhecimentos nesta área tão recente pode trazer vários benefícios educacionais e computacionais. Um exemplo está no que se refere à motivação de alunos e professores, que pode ser incentivada com a construção de trabalhos/projetos interdisciplinares, construindo conhecimentos e avaliações unificadas entre as disciplinas envolvidas. Tal aplicação pode gerar, tanto no caso da computação quanto das disciplinas específicas agrupadas, um aliamento de teoria e prática e a contextualização com aplicações práticas aplicadas ao cotidiano.
	
    
    É necessária uma análise crítica sobre o curso, atentando principalmente, para o fato de estar ou não ajustado às expectativas e necessidades da atual sociedade, da educação, da computação e visando futuras transformações sociais. É importante o engajamento de pessoas estudando esta área ainda recente, visando melhorias constantes em todos os âmbitos do curso. Ações como o acompanhamento contínuo dos planos de ensino, bem como sua aplicação prática merecem especial atenção. Assim seria de fato incentivado o aumento do número de professores, com motivação e qualificação indispensáveis.
 
 
 
%%%%%%%%%%%%%%%%%%%%%%%%%%%%%%%%%%%%%%%%%%%%%%%%%%%%%%%%%%%%%%%%%%%%%%%%%%%%%%%%
%%%%%%%%%%%%%%%%%%%%%%%%%%%%%%%%%%%%%%%%%%%%%%%%%%%%%%%%%%%%%%%%%%%%%%%%%%%%%%%%
%%%%%%%%%%%%%%%%%%%%%%%%%%%%%%%%%%%%%%%%%%%%%%%%%%%%%%%%%%%%%%%%%%%%%%%%%%%%%%%%
\section{Parte prática do curso - primeiros estágios docentes}%
 
	Uma parte que conta significativamente para a formação de professores de computação e para a construção da identidade docente nos alunos de LC diz respeito aos estágios supervisionados que fazem parte do currículo de referência do curso (SBC 20XX). Para maioria dos alunos trata-se da primeira experiência como docente, o que é um grande desafio.
	
    
    Neste curso é fundamental que a formação acadêmica seja compatível com a realidade prática. Assim, as experiências de estágios docentes para os alunos da LC mostram-se fundamentais para a noção dos alunos da real experiência docente. Neste processo há uma grande reflexão crítica sobre a prática pedagógica por parte dos futuros docentes, além uma visibilidade da real intervenção que estes causam na realidade local/regional. Matos (2013) evidencia a importância da articulação entre teoria e prática, contextualizando o estudante para poder apreciar, no campo da prática, "elementos da sua (futura) realidade enquanto profissional".
	
    
    Os alunos passam por uma grande experiência de crescimento profissional, pessoal e acadêmico. É a experiência da prática docente e o desafio da transposição de uma linguagem científica e técnica, para uma forma mais acessível aos estudantes bem como a contextualização e significação dos conhecimentos. Trata-se do aprendizado de competências próprias da atividade profissional e à contextualização curricular, objetivando o desenvolvimento do estudante para a vida cidadã e para o trabalho (CRUZ 2013). 
	
    
    As primeiras experiências como professores fornecem aos licenciandos a oportunidade de conhecer e questionar a estrutura escolar, o funcionamento e as atividades desenvolvidas neste ambiente bem. Tal experiência favorece o desenvolvimento das práticas interdisciplinares na Educação Básica, além de aumentar o interesse dos alunos das licenciaturas pelo trabalho docente. Gera-se a aquisição de experiências práticas docentes e uma busca por melhorias do processo ensino-aprendizagem. Os alunos passam pelo desenvolvimento da visão crítica do ambiente e do universo em que se situa a sua profissão, bem como por uma iniciação de inserção na atividade profissional. Há uma aproximação entre os estudantes e os setores econômicos, educacionais e culturais que demandam a sua atividade profissional.
	
    
    Os licenciandos tornam-se mais qualificados nas atividades docentes de ensino da computação, bem como na possibilidade de criação, projeto e construção de software educacional com objetivo de melhorar o processo e a gestão educacional. Há a integração entre os conhecimentos tecnológicos e didáticos, oferecendo aos docentes a oportunidade de adquirir conhecimentos em Tecnologia da Informação e as formas de introduzi-los no contexto educacional inovando o processo de ensino de sua disciplina. As experiências docentes em estágios práticos são fundamentais para uma formação sólida e abrangente como educadores, com base nas áreas técnicas de computação e com ênfase nos aspectos pedagógicos e sociais existentes na realidade das escolas.
	
    
    Cambraia e Scaico (20XX) valorizam os primeiros momentos de experiência docente pela aquisição de qualidades e saberes fundamentais à prática docente, como: saber se  aproximar dos estudantes;  identificar o perfil dos alunos, sua familiaridade com tecnologias, predileções e hábitos; e situar-se no contexto social em que se encontra, seus fatores culturais e de infraestrutura que influenciarão em suas práticas. O estágio demanda consistência no exercício para a formação, enquanto a formação universitária deverá garantir a constituição das competências objetivadas para a Educação Básica (CRUZ 2013).	
	
    
    Junior (et al, 2009) em suas pesquisas encontram diversas contribuições que o estágio docente trouxe à formação pessoal e profissional de alunos de licenciaturas. A segurança e organização dos conteúdos, segurança de falar em público e como falar, a autonomia e postura profissional, além de uma vivência real de situações de trabalho na prática que contribuem com a formação docente. Os autores concluem em sua pesquisa que tal etapa da formação de professores contribui intensamente com a desenvoltura dos futuros professores nas áreas de planejamento, ensino e avaliação. 
	
    
    Planejamento neste contexto é tido como a organização das disciplinas em tópicos a ser ministrada, elaboração dos exercícios e atividades a serem desenvolvidos pelos estudantes, organização e pesquisa do material didático e confecção dos planos de aulas a serem desenvolvidos junto com os estudantes no processo de ensino e aprendizagem. Algumas dificuldades que os autores apontam como recorrentes em professores que estão iniciando sua prática docente dizem respeito a:
    \begin{itemize}
	\item planejamento do plano da disciplina: elaborar o planos de ensino e planos de aula; e
	\item dividir o conteúdo de cada aula, elaborando sua sequência de aulas durante todo um semestre e cuidando para não haver conteúdo não ministrado, além de aulas com qualidade.
	Neste contexto o ensino corresponde à execução do planejamento previamente elaborado e a capacidade de adaptação do planejamento de acordo com o contexto em que o professor se encontra. Dificuldades recorrentes apontadas pelos autores quanto ao ensino são:
	\item dificuldade de falar em público; 
	\item aproximação e contato direto com os alunos, além de compreender bem o contexto em que estão integrados;
	\item dificuldade em mudar o plano pedagógico já estabelecido, mesmo quando ao chegarem no momento da prática as condições encontradas não serem as esperadas (falta de infraestrutura, menos ou mais qualificação dos alunos do que o esperado, etc.)
    \end{itemize}
	
    
    Quanto a avaliação os autores perceberam uma tendência dos professores avaliarem com trabalhos práticos, seminários ou provas, pois "foi assim que foram avaliados enquanto cursavam toda sua formação. O docente tende a reproduzir as formas que ele mesmo foi avaliado e ensinado, por mais que tais métodos sejam tão criticados, se não tiver como referência um pensamento crítico e reflexivo sobre suas práticas, questionando-se acerca de seus métodos e sua efetividade, sempre lembrando que está ali para construir ambientes favoráveis à aprendizagem junto com os alunos.  Durante estágios práticos acompanhados e com valores críticos, os estudantes podem adquirir diversas técnicas pedagógicas de avaliação usando a criatividade e a inovação como bases, ao invés de apenas reproduzir o modelo educacional atual.
	
    
    Os licenciando, nestas práticas, fazem um constante exercício de transposição didática ao criar planos de aula, softwares educacionais, situações de estudos e projetos interdisciplinares. Exercício este de fundamental importância na atividade docente. Cambraia e Scaico defendem que para que a transposição pedagógica ocorra com menos distorções, a ampliação de espaços de formação de professores, como a proporcionada pelos estágios docentes se faz necessário, já que os estudantes têm contato permanente com professores da universidade, professores da educação básica, com alunos e suas realidades, tendo a possibilidade de vivenciar ações e potencializar reflexões em imersão nessa coletividade. Nessa interlocução, trazem contribuições para repensar as licenciaturas e também percebem que o conhecimento que elaboram na universidade precisa ser significado no contexto em que se inserem, produzindo um “saber escolar”.
 
	Pimenta (2004) defende o estágio como o 

(((((citacao menor à direita))))))
\begin{quotation} 
[...]lócus onde a identidade profissional é gerada, construída e referida. Tal experiência está voltada para o desenvolvimento de uma ação vivenciada, reflexiva e crítica. [...]. O estágio supervisionado é a aproximação da realidade da sala de aula, da Instituição de ensino, e a importância disso para a reflexão teórica sobre a prática aplicada é fundamental.
\end{quotation}
 
Matos (2013) partilha de visão semelhante, afirmando que o contato com o campo da prática pode contribuir para a construção da identidade docente de estudantes de cursos de licenciatura. Dado que esse contato faz parte do percurso de formação, o currículo tem forte influência na concepção da identidade.


	Assim, as primeiras experiências em práticas docentes mostram-se fundamentais para o estudante de Licenciatura em Computação, onde será possível um aliamento da teoria aprendida e da prática profissional esperada, diante de uma contextualização profissional e social para formar o pensamento crítico e reflexivo do professor. Tal experiência colaborando com a formação da identidade docente no aluno da licenciatura, e contribuindo com a compreensão da profissão a ser exercida em uma atividade prática. O aprendizado com professores mais antigos pode contribuir com a inserção e ambientalização do estudante, formando um trabalho contextualizado. Assim, preparando o futuro professor para os desafios que irá ter de enfrentar, e já visualizando soluções de forma criativa e inovadora.