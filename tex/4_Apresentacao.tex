%%%%%%%%%%%%%%%%%%%%%%%%%%%%%%%%%%%%%%%%%%%%%%%%%%%%%%%%%%%%%%%%%%%%%%%%%%%%%%%%
%%%%%%%%%%%%%%%%%%%%%%%%%%%%%%%%%%%%%%%%%%%%%%%%%%%%%%%%%%%%%%%%%%%%%%%%%%%%%%%%
%%%%%%%%%%%%%%%%%%%%%%%%%%%%%%%%%%%%%%%%%%%%%%%%%%%%%%%%%%%%%%%%%%%%%%%%%%%%%%%%
\section{Formação de professores}%


	A investigação é muito importante. O currículo e os programas de ensino são muito importantes. A gestão e administração das escolas são muito importantes. Os materiais didáticos e as tecnologias são muito importantes. Mas nada substitui um bom professor (NÓVOA, 2011).
 
	
	O tema formação de professores possui extrema relevância em ser debatido, devido ao papel fundamental do professor em todo o sistema de ensino e, consequentemente, na transformação e evolução política, econômica e social da nação. Este tema abrange três diferentes dimensões no âmbito educacional, quais sejam:
	* Professor em sala de aula:  formas de ensino-aprendizagem; métodos e técnicas de ensino; didática; relação aluno-professor; utilização de TICs no ensino; realização de aulas, planejamentos e avaliações.
	* Planejamento dos cursos de formação docente: planejamento educacional; currículo dos cursos de licenciaturas e pedagogia; organização dos cursos nas Universidades; estágios obrigatórios; âmbito institucional.
	* Políticas públicas educacionais: dimensão política; políticas públicas; planos de carreira; incentivo às pesquisas na área docente; valorização dos professores; âmbito governamental.
 
[[ IDENTIFICAR UMA DIMENSÃO A QUE EU ME PROPONHO INVESTIGAR MAIS ? ]]
 
	NÓVOA (2009) aponta que a formação docente adota princípios que estão muitas vezes afastados da realidade profissional. O autor aponta para a necessidade de uma formação de professores dentro da realidade da profissão, buscando a aproximação da formação docente com a atuação profissional. É necessário olhar para a dimensão pessoal e profissional na produção da identidade do professor, gerando uma "profissionalidade docente que não pode deixar de ser construída no interior de uma pessoalidade do professor".
    
    
	Para isso, Nóvoa aponta cinco pontos fundamentais para uma formação docente de qualidade:
    \begin{itemize}  
	\item Prática / praxis: O professor deve saber bem aquilo o que irá ensinar e evitar a todo custo a dicotomia entre teoria e prática. A formação docente deve conter uma componente práxica centrada na aprendizagem dos alunos e em estudos de caso concretos. Buscar soluções de casos concretos envolve ações práticas bem como análises que mobilizam conhecimentos teóricos, logo a dicotomia teoria/prática no âmbito educacional prejudica a atuação docente. A educação visa ações, e não puramente o saber, sendo necessária uma prática profissional para reflexão e formação. É fundamental abandonar a ideia de que ser professor é igual a capacidade de transmitir o saber, e reconhecer que a inovação é elemento fundamental para o processo de formação. Nas palavras de Freire (2011) "a reflexão crítica sobre a prática se torna uma exigência da relação Teoria/Prática sem a qual a teoria pode virar blá-blá-blá e a prática, ativismo". 
    
    
	\item Cultura profissional: a formação docente deve ser baseada na aquisição da cultura profissional, onde os professores mais experientes possuem um papel fundamental na formação dos professores mais novos. Tanto a aquisição de conhecimentos teóricos/metodológicos quanto de conhecimentos práticos é parte fundamental da formação docente. O aprendizado com professores mais experientes, o registro das práticas para reflexões, a compreensão dos sentidos da instituição escolar e o exercício da avaliação são componentes fundamentais da formação docente.
    
    
	\item Tato pedagógico: no papel de professor, é impossível separar a dimensão profissional da dimensão pessoal. A profissão docente requer uma interação social que envolve a capacidade de relação e de comunicação com os discentes. Os professores devem estar preparados para o trabalho de si próprios, a auto-reflexão e a autoanálise visando melhorar a sua prática pedagógica. É fundamental a compreensão de que o professor não se resume a parte técnica e científica. Os alunos trazem diferentes realidades sociais e culturais para a sala de aula, trazendo uma dimensão humana e relacional ao ensino. É interessante que os professores trabalhem desde a formação docente a construção de narrativas sobre sua história de vida pessoal e profissional para elaborar o auto-conhecimento dentro do conhecimento profissional. Trabalhar o registro de suas vivências pessoais e profissionais para a maior consciência de seu trabalho e identidade através da reflexão.
    
    
	\item Trabalho em equipe: a atual complexidade do trabalho escolar requer a atuação de equipes pedagógicas, uma vez que a competência coletiva é mais do que o somatório das competências individuais. A escola é um local fundamental na formação de professores, e a análise compartilhada de práticas docentes pode servir como rotina de reflexão do trabalho docente. A experiência coletiva pode formar o conhecimento profissional, bem como a ética profissional através do diálogo entre professores. Os processos de mudança podem ser transformados em práticas concretas de intervenção.
    
    
	\item Responsabilidade social: a formação docente deve levar em consideração o fator social da educação com a transmissão de princípios, valores, inclusão social, compreensão da diversidade cultural e o rompimento das fronteiras e barreiras para com os estudantes. A profissão docente gera uma direta intervenção no espaço público através da comunicação da escolar com o exterior e do papel social do professor.  
    
    \end(itemize)
    
	Uma vez que o ensino é uma ação humana, este tem passado por diversas mudanças oriundas das variáveis sociais, econômicas e políticas que influenciam, direta ou indiretamente a educação. Estão surgindo novos conhecimentos, práticas e técnicas, bem como espaços para ensinar e aprender, para construir e compartilhar conhecimento. Novas oportunidades de posturas metodológicas interdisciplinares têm espaço na busca pela integração curricular como mecanismo de desenvolvimento de habilidades e competências (MATOS 2013). A formação e a profissionalização docente constituem um processo contínuo e inacabado, sempre em movimento.
    
    
	Segundo CRUZ (2013), atualmente há o estreitamento dos vínculos entre o setor educacional e o mundo do trabalho devido às inúmeras transformações científicas e tecnológicas atuais,  bem como a integração mais intensa dos mercados.
    
    
	GATTI (1997) constata que no atual cenário sócio-político, a informação e a comunicação ocupam um papel central no movimento de transformação social. Em plena era da comunicação, nada é mais essencial do que decodificar e interpretar informações, capacidades estas que dependem de um domínio cultural de diferentes áreas de saberes, referenciando diretamente à educação. A questão de formação de professores é um desafio para as políticas públicas, visando uma democratização da distribuição do conhecimento sistematizado a parcelas mais amplas da população. No âmbito das políticas públicas a atividade docente tem perdido a atratividade cada vez mais, por diversos motivos que envolvem o prestígio social e a qualidade da formação (LOUZANO et al 2010).	
    
    
	Além da valorização docente, em uma sociedade que necessita transmitir uma quantidade cada vez maior de conhecimentos a seus cidadãos, a formação do professor não pode estar voltada para a simples acumulação e transmissão de conhecimentos, mas sim em aliar as teorias conhecidas com a prática profissional, a fim de causar uma real e efetiva intervenção social. A atividade do professor não é apenas do professor para consigo mesmo, mas de si para os alunos e, consequentemente, para a sociedade. Os professores atuam como orientadores e problematizadores que podem construir relações entre o conhecimento cotidiano dos alunos e o conhecimento científico, trazendo mais interesse ao processo educativo ao valorizar aquilo que os alunos já sabem. Piaget (1974) colabora com esta ideia, reforçando que a aprendizagem não parte do zero, mas utiliza conhecimentos que os alunos trazem consigo, Os professores não devem se portar como os detentores do conhecimento que estão ali apenas para transmitir informações, mas proporcionar um ambiente em que o conhecimento possa ser construído em conjunto, incentivando os alunos a realizarem as atividades propostas de maneira crítica e reflexiva, sempre valorizando a curiosidade.
    
    
	Torna-se fundamental a reflexão sobre a formação inicial docente. O conhecimento técnico é importante na formação de professores, mas não suficiente para lidar com um contexto humano e social de diversidade. Um professor com teorias do conhecimento mais elaboradas não tem necessariamente uma prática docente mais efetiva. Este processo requer que sejam abordados temas que conduzam o professor para a compreensão do processo de ensino-aprendizagem, levando-o à reflexão sobre conhecimentos de elementos favoráveis para provocar os alunos a adquirirem autonomia na solução de problemas, a incentivar o raciocínio crítico-reflexivo, bem como incentivar a "curiosidade epistemológica" dos alunos. O professor deve ser um profissional que analise constantemente as propostas de ensino que faz aos seus alunos e que se auto-avalia permanentemente. A formação docente deve incentivar a constituição do professor-pesquisador desde a formação inicial, bem como primar por semear nos professores a atitude reflexiva,  incentivando o professor a pensar suas ações e buscar novas soluções e caminhos para a sua prática (FREIRE 2011). Segundo Vasconcelos (2012), o docente deve ter a percepção crítica de saber quando deve deixar de lado as técnicas e métodos já estabelecidos, quando as diversas situações sociais que podem ocorrer em sala de aula assim exigirem, e inovar com a criação de novas técnicas e transformações metodológicas de acordo com o contexto social em que se encontra.	
    
    
	////""Cruz (2013) pontua a resolução CNE/CP no 1, de 18 de fevereiro de 2002, que regulamenta os cursos de formação de professores, especialmente no que se refere à aprendizagem do estudante, trato da diversidade, enriquecimento cultural, práticas investigativas, projetos de desenvolvimento dos conteúdos curriculares, o uso de tecnologias da informação e da comunicação e de metodologias, estratégias e materiais de apoio inovadores, além de hábitos de colaboração e de trabalho em equipe.""////
    
    
	Matos (2013) reforça a observação para os atuais curriculos de formação docente, "tradicionalmente lineares e baseados no conteudismo modular", analisando que tais currículos já não satisfazem a sociedade atual formada por pessoas interconectadas a diversos tipos de mídias. As transformações tecnológicas e sociais interferem diretamente na formação do professor que, "educado dentro de uma concepção curricular linear, terá como campo da prática um universo que exige interação dialética e transversalidade". A postura docente vem exigindo que os docentes repensem seus métodos e posturas filosóficas em virtude das intensas transformações tecnológicas, culturais e sociais que vêm ocorrendo na atualidade.	
    
    
	Os currículos de formação docente, além de fomentar a transformação da educação pela revolução da sociedade, também devem abordar sobre como se dá a construção da identidade do professor durante o curso, identidade esta que pode ser identificada a partir dos fenômenos relacionados ao sentimento de "pertença dos indivíduos à categoria docente" (MATOS 2013). Os estudantes de licenciaturas trazem consigo suas experiências de todo seu processo escolar, mostrando que a formação da identidade do sujeito com a docência é iniciada antes de sua entrada em um curso de graduação. O estabelecimento da identidade do professor é constituída ao longo de sua trajetória profissional, no entanto em seu processo de formação são consolidadas as opções e intenções da profissão que o curso se propõe legitimar. Além das disciplinas específicas que os alunos vivenciam, os estágios docentes iniciais são experiências fundamentais onde a identidade profissional é gerada, construída e referida. Trata-se do desenvolvimento de uma ação vivenciada, reflexiva e crítica e, por isso, deve ser planejado gradativa e sistematicamente com essa finalidade. (PIMENTA, 2006, p. 26).
