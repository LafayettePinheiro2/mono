%%%%%%%%%%%%%%%%%%%%%%%%%%%%%%%%%%%%%%%%%%%%%%%%%%%%%%%%%%%%%%%%%%%%%%%%%%%%%%%%
%%%%%%%%%%%%%%%%%%%%%%%%%%%%%%%%%%%%%%%%%%%%%%%%%%%%%%%%%%%%%%%%%%%%%%%%%%%%%%%%
%%%%%%%%%%%%%%%%%%%%%%%%%%%%%%%%%%%%%%%%%%%%%%%%%%%%%%%%%%%%%%%%%%%%%%%%%%%%%%%%
\section{TICs na educação}%
 


3.0.1 - Tecnologias da Informação e Comunicação (TICs) na educação
		- tic + educacao
		- ensino de computação 
		- pensamento computacional
 
[TIC + EDUCACAO]
	
	Do mesmo modo que não é o objeto que leva à compreensão, não é o computador que permite ao aluno entender ou não um determinado conceito. A compreensão é fruto de como o computador é utilizado e de como o aluno está sendo desafiado na atividade de uso desse recurso. (VALENTE, 1999, p. 37)
 
	As influências das mudanças sociais, econômicas e políticas sobre a educação, têm causado grandes mudanças no ensino. A sociedade atual encontra-se altamente informatizada, e os conhecimentos sobre recursos tecnológicos já são imprescindíveis para a vida social e profissional.  (MATOS, 2013) ressalta que a necessidade das habilidades para o mundo do trabalho no século XXI passam pelas tecnologias de informação e comunicação (TIC). Estas têm ampliado as possibilidades de comunicação e interação entre as pessoas e entre as disciplinas e áreas de conhecimento, contribuindo com a busca por soluções integradoras e desenvolvimento de competências necessárias para a atual realidade. As TIC podem contribuir com o processo educativo, possibilitando aos educadores recursos para promoção do diálogo e da colaboração, permitindo a busca por abordagens educacionais "interdisciplinares, multidisciplinares e pluridisciplinares", numa relação dialética entre tecnologia e educação.
    
    
	Estão surgindo novos conhecimentos, práticas, técnicas e espaços para ensinar e aprender, colaborando com a construção e compartilhamento de conhecimento. A educação mostra oportunidades de posturas metodológicas interdisciplinares na busca pela integração de disciplinas como forma para desenvolvimento de habilidades e competências (LOPES, 2008 ---> falta referência --> no artigo do matos dá para achar talvez).
    
    
	As TIC possibilitam aos educadores recursos para a integração de novas formas e ensino-aprendizagem. As novas gerações possuem grande interesse nas novas tecnologias que vêm sendo desenvolvidas. Aliar o interesse já existente com os conteúdos a serem ministrados pode ser uma estratégia fundamental na transformação da educação. Diversos recursos vêm sendo desenvolvidos, contribuindo fortemente para a evolução do ensino, tais como aplicativos e softwares educacionais, plataformas de ensino à distância, cursos online, etc. Tais recursos podem fomentar alternativas metodológicas interdisciplinares assim como constituir formas de ensino onde o aluno participa ativamente na construção de seu saber e na investigação de suas curiosidades e críticas, ao invés de ser apenas um personagem passivo em seu processo de formação. 	A distância foi eliminada pela interação no ciberespaço e este é um fator que também pode ser utilizado a favor do processo educacional. 
    
    
	Com o grande desenvolvimento computacional atual, a formação de professores deve estar atualizada para lidar com a informática em sala de aula. A tecnologia já está presente nas escolas. Trazida tanto pelo governo, através de da implantação nas escolas de laboratórios de computação, Internet e diversos recursos tecnológicos, quanto pelos próprios alunos, que trazem seus celulares com acesso a Internet e uma série de outros recursos para a escola. A integração da tecnologia na educação pode seguir uma linha de informática educativa, que pode dar autonomia aos alunos para pesquisar suas curiosidades e indagações e trazer informações para somar à sala de aula. A tecnologia já não pode ser ignorada em uma sociedade majoritariamente informatizada, e pode ser integrada de formas úteis para o ensino e a educação. Para isso, é necessária uma mudança na mentalidade dos professores e a própria capacitação para lidar com as tecnologias recentes e com as estratégias de integração de tais tecnologias em sua forma de ensinar.
    
    
	A interação dos alunos com a tecnologia pode gerar a construção de conhecimentos através de diversos objetos educacionais e várias fontes de pesquisa. A instrução escolar no uso tecnológico gera um usuário mais competente e consciente do que está fazendo na máquina e na rede, além da elaboração de um senso crítico das informações recebidas. Entretanto, tal prática depende de professores dispostos a alterar suas práticas pedagógicas e investir na inovação, no desafio ao novo. Também é necessário que sejam oferecidas condições favoráveis a estes professores para que ponham isto em prática, como capacitação técnica, laboratórios com boa infraestrutura, bons equipamentos, Internet, etc.
    
    
	A inserção da informática na educação requer que os profissionais utilizem os recursos com eficiência e para isso é fundamental a capacitação na formação docente para lidar com as TIC visando a melhora da qualidade do ensino. ALMEIDA (2012) ressalta que o uso das tecnologias e a reestruturação das práticas educativas podem instituir mudanças significativas na forma de dimensionar os conteúdos curriculares e as formas de acesso às informações que serão trabalhadas. Assim, a integração das TICs em sala de aula representa alterações didático-curricular que podem influenciar fortemente a proposta educacional.	
    
    
	A implantação da tecnologia em sala de aula é um processo desafiador. Tanto pelas mudanças nas práticas pedagógicas e na visão do processo educacional por parte dos educadores, quato por possíveis dificuldades enfrentados pelos profissionais que buscam integrar a computação em suas disciplinas, especialmente na rede pública, tais como:
    	\begin(itemize)
		\item Falta de equipamentos ou baixa qualidade dos mesmos;
		\item Falta de infraestrutura;
		\item Formação docente deficiente na atuação com TICs; e
		\item Falta de capacitação básica dos alunos.
        \end(itemize)
        
	 A integração das TIC com o processo educacional envolve uma postura pedagógica mais aberta à contribuição do outro, onde o professor valoriza o aluno como um personagem ativo na construção dos saberes, coletivamente. Ao invés da pura transmissao verbal dos conhecimentos em sua forma tradicional, a integração com TIC tende a tornar o processo educacional mais interativo para os alunos, proporcionar-lhes diversos desafios. Integrando softwares educacionais, Internet, vídeos, editores de texto e outros objetos pedagógicos, é possível proporcionar mais pesquisa, interação, análise e reflexão dos alunos a respeito dos conteúdos propostos, buscando eliminar a visão do aluno como um receptor passivo para ser um personagem ativo na construção e produção de saberes.
     
     
	As TIC podem ser ferramentas eficazes de linguagem e representação do conhecimento e da comunicação, favorecendo a implantação de atividades interdisciplinares entre computação e outras áreas do saber. Assim fica proposto um ambiente construtivista, onde o aluno pode construir conhecimentos e aprendizados auxiliado por recursos computacionais, além de buscar informações com facilidade. Faz-se necessária a criação de ações criativas e inovadoras para despertar o interesse da comunidade escolar, rodeada muitas vezes por uma realidade problemática.
    
    
	Cruz (2013) ressalta a preocupação que se faz presente no que diz respeito à formação do futuro professor para o trabalho com tecnologias, que está snedo amplamente cobrado durante a atuação escolar como docente, mas não obrigatório nos cursos de Licenciaturas. O autor defende que todas as Licenciaturas deveriam, obrigatoriamente, abordar disciplinas sobre informática aplicada à educação e similares. A abordagem das tcnologias nas Licenciaturas em geral deve primar pela capacitação e também pela diferenciação na abordagem utilitária de softwares e o tratamento do assunto no contexto da Informática na Educação.
    
    
%%%%%%%%%%%%%%%%%%%%%%%%%%%%%%%%%%%%%%%%%%%%%%%%%%%%%%%%%%%%%%%%%%%%%%%%%%%%%%%%
%%%%%%%%%%%%%%%%%%%%%%%%%%%%%%%%%%%%%%%%%%%%%%%%%%%%%%%%%%%%%%%%%%%%%%%%%%%%%%%%
%%%%%%%%%%%%%%%%%%%%%%%%%%%%%%%%%%%%%%%%%%%%%%%%%%%%%%%%%%%%%%%%%%%%%%%%%%%%%%%%
\section{Ensino de Computação}%
 
[Ensino de Computação]
	Diversos autores apontam sobre a necessidade de integrar conhecimentos computacionais na educação básica, atitude que já vem sendo tomada por diversos países desenvolvidos. Castro (2013) traz questionamentos pertinentes sobre esta temática: porque ensinar Computação nas escolas? Como os conteúdos deveriam ser ensinados? quais tópicos deveriam ser lecionados? e para quem esta educação poderia ser significativa?
    
    
	As diversas mudanças sociais trazidas pelo desenvolvimento tecnológico traz à tona a necessidade de oferecer uma educação que proporcione aos alunos a habilidade de trabalhar com diferentes conhecimentos produzidos pela Computação e de explorá-los a seu favor não apenas no campo da diversão e do entretenimento (CASTRO 2013b).
    
    
	Além disso, a inserção da informática no currículo escolar contribui com a utilização do computador como um instrumento de apoio às matérias escolares e aos conteúdos relacionados, além de preparar os discentes para uma sociedade cada vez mais informatizada. As Instituições de Ensino não podem ficar fora desta realidade levando em conta o quanto as novas tecnologias têm modificado as relações do homem com o mundo. É necessário reconhecer o potencial das TIC, não apenas como mídia educacional ou como comunicação, mas também como recurso base para o desenvolvimento de materiais de apoio à educação e às demais áreas do conhecimento.	
 

%%%%%%%%%%%%%%%%%%%%%%%%%%%%%%%%%%%%%%%%%%%%%%%%%%%%%%%%%%%%%%%%%%%%%%%%%%%%%%%%
%%%%%%%%%%%%%%%%%%%%%%%%%%%%%%%%%%%%%%%%%%%%%%%%%%%%%%%%%%%%%%%%%%%%%%%%%%%%%%%%
%%%%%%%%%%%%%%%%%%%%%%%%%%%%%%%%%%%%%%%%%%%%%%%%%%%%%%%%%%%%%%%%%%%%%%%%%%%%%%%%
\section{Pensamento educacional}%
 
[PENSAMENTO COMPUTACIONAL --> contribui com educação em computação]
	Conforme Matos e Silva (20XX), a inserção da Computação na formação do indivíduo, na educação básica ou superior, não visaria apenas de formar indivíduos capazes de compreender a máquina, mas também de criar um pensamento computacional. Wing (2006, p. 33) afirma que com tal habilidade o indivíduo seria capaz de “resolver problemas, desenvolver sistemas e compreender o comportamento humano, recorrendo aos conceitos fundamentais para a Ciência da Computação”.
    
    
	Castro (2013b) defende o uso das TIC em sala de aula como uma fuga da instrumentalização pura e da inserção do pensamento computacional na Educação, e Nunes (2011) afirma que a introdução de tal habilidade na educação básica provê os recursos cognitivos necessários à resolução de problemas, transversal a todas as áreas do conhecimento.
    
    
	 Desta forma, a integração de conhecimento em Computação na educação básica colabora com a autonomia para a resolução de problemas, incentivando o desenvolvimento da lógica de pensamento, pensar e agir no sentido de como resolver problemas e quais equipamentos (computadores) corretos utilizar para esta resolução, desenvolvendo algoritmos que podem ser expressos em diferentes níveis de abstração.
     
     
	A efetiva integração da computação na educação básica depende de fatores que ajudem os alunos a entenderem o conteúdo proposto com fluidez, por serem parte de um paradigma um pouco diferente do estudado convencionalmente. A linguagem utilizada para a comunicação com esses estudantes é fundamental para que se possa estabelecer o interesse e a motivação necessários para as ações educacionais. Uma linguagem puramente técnica ou uma linguagem mais informal mas sem exatidão, não seriam efetivas no processo de ensino-aprendizagem.
    
    
	Além da linguagem utilizada, a inserção de conceitos técnicos como representação da informação (texto e imagem), criptografia, ordenação, busca, redes de computadores, compressão de dados, deve ser feita através de situações do cotidiano dos alunos, facilitando o aprendizado de áreas muitas vezes completamente novas para os alunos. Tal abordagem tende à mudança de visão de que estes conceitos sejam tão difíceis, ao serem abstraídos para situações do cotidiano dos estudantes,
    
    
	A introdução do pensamento computacional na escola é um  movimento em construção. Atualmente tem aparecido como o ensino de uma linguagem de programação (CAMBRAIA e SCAICO, 2013). Não se trata de uma técnica específica, mas de uma forma de organização do pensamento e de resolução de problemas, que conforme CASTRO (2013) pode ser trabalhado de forma interdisciplinar, rompendo com a tendência de manter uma "elaboração linear e fragmentada de conteúdos", e contemplando as inter-relações de conhecimentos.
    
    
	O desenvolvimento de habilidades computacionais na Educação Básica é importante intelectualmente, levando à possibilidade de múltiplos caminhos profissionais futuros, desenvolvendo a capacidades de resolução de problemas e motivando os estudantes ao desenvolvimento da criatividade e da autonomia. Além disso, contribui no intuito de formar cidadãos com competências e habilidades necessárias para conviver e prosperar em um mundo cada vez mais tecnológico e global, contribuindo com o desenvolvimento social e econômico do País.
    
    
	Bezerra e Silveira (20XX) trazem um conjunto de conceitos, habilidades e competências computacionais que podem ser esperados de um aluno da Educação Básica na sociedade atual, conforme a imagem abaixo. Os autores reforçam a necessidade de formação de educadores capacitados para atuar nesse nível de ensino proporcionando o desenvolvimento tais habilidades e competências, aos alunos.
 
 
Figura 2. Conceitos, Habilidades e Competências da Ciência da Computação para Educação Básica Fonte: Bezerra e Silveira 20XX
 
	Desta forma,  o binômio Computação-Educação pode fornecer reais contribuições para o país através de um crescimento qualitativo e quantitativo no desenvolvimento tecnológico, na revolução educacional e no desenvolvimento das tecnologias educacionais.




