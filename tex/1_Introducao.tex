

Com o constante desenvolvimento tecnológico, os recursos computacionais entraram de forma definitiva no cotidiano da sociedade. Atualmente encontram-se informatizadas as empresas, o meio acadêmico e os domicílios, notando-se o amplo uso dos recursos computacionais nas
tarefas cotidianas de grande parte da sociedade. 

Este desenvolvimento tecnológico vem gerando diversas transformações sociais. O número de informações disponíveis às pessoas está aumentando a cada dia, bem como as formas de circulação destas informações. \citep{quim2013} nos aponta o quanto o espaço de mudanças sociais através da evolução tecnológica é amplo. A informatização da sociedade requer cidadãos mais qualificados tecnologicamente para  uso eficaz das Tecnologias de Informação de Comunicação (TIC) na vida profissional e social. As empresas estão investindo fortemente em qualificação tecnológica de seus funcionários para as suas demandas; a informática já se encontra altamente integrada à Educação e à comunicação interpessoal. A sociedade em geral vem necessitando de um acesso inicial aos recursos e à capacitação efetiva, buscando desenvolver um senso crítico dos usuários para saber utilizar de forma consciente e reflexiva os recursos disponíveis, além de estarem cientes de possíveis vulnerabilidades e formas de evitá-las.


A evolução tecnológica é um fenômeno relativamente recente, e mais recente ainda é o debate sobre o ensino de Computação. \citet{castro2013a} questionam pontualmenta que, se há a necessidade de que um número crescente de pessoas estejam qualificadas para o uso de tais tecnologias, quem lhes ensinará? Este aprendizado poderá ocorrer eficazmente de forma autônoma? É necessário um professor de Computação para auxiliar a sociedade na integração eficaz da tecnologia em suas tarefas? Como superar preconceitos e consolidar uma formação docente em computação de qualidade frente a tantas variáveis? Enfim, quais são os papéis de uma Licenciatura em Computação (LC) no 
contexto educacional brasileiro?


Segundo \citet{gatti97}, em plena era da informação e comunicação, é exigido dos cidadãos uma capacidade cada vez maior de busca, compreensão, apreensão e interpretação de informações de forma rápida e eficaz, mudando a forma com que o fenômeno ensino-aprendizagem vem ocorrendo. Um grande número de informações está disponível às pessoas, e o grande desafio atual é saber buscar e filtrar estas informações com eficácia.


Frente a tamanha demanda de ensino Computacional, entra em debate o curso de Licenciatura em Computação (LC). Ao surgir, a proposta incial do curso era o ensino de computação na educação básica e no ensino profissionalizante. Contudo, com a forte integração tecnológica em diversos segmentos da sociedade, várias outras possibilidades de atuação profissional surgem para o licenciado em computação a cada dia, colaborando com o desenvolvimento da Educação e da Computação na sociedade \citep{castro2013b}. 


A Licenciatura em Computação possui um caráter inovador e desafiador ao integrar em seu currículo tecnologia, educação e ciência. A Computação possui mudanças diárias e aceleradas enquanto a Educação é formulada de maneira histórica, com transformações mais vagarosas. \citet{matos2013a} nos mostra o quão grande é o desafio de unir áreas tão distintas, mas que podem colaborar fortemente uma com a outra. A interdisciplinaridade é uma característica marcante deste curso, formando profissionais habilitados para o trato tecnológico, científico e humanístico.


\citet{cabral08} esclarece que ainda não existe uma regulamentação específica para o trabalho do Licenciado em Computação. O egresso pode ser professor de computação no ensino básico e técnico, bem como atuar em inúmeros espaços educativos, que vão além da sala de aula tradicional, sem descaracterizar a sua função primordial de profissional da educação em computação. \citet{castro2013a} nos mostra que tais espaços oferecem diversas oportunidades de atuação em práticas educativas por meio de novas Tecnologias da Informação e Comunicação. Há falta ou abundância de espaço de atuação profissional para o profissional de educação em computação? Faz-se necessária uma real compreensão das formas de atuação deste profissional e sua forte colaboração para a sociedade e para o desenvolvimento tecnológico. 

Há uma real necessidade de estudos nesta área, ainda carente de pesquisa científica, para que seja possível definir estratégias de atuação político-científicas sobre as propostas curriculares e possíveis articulações dos grupos de pesquisa em Computação com outras comunidades epistêmicas, sobretudo a comunidade de pesquisa em Educação \citep{matos2013a}.


%%%%%%%%%%%%%%%%%%%%%%%%%%%%%%%%%%%%%%%%%%%%%%%%%%%%%%%%%%%%%%%%%%%%%%%%%%%%%%%%
%%%%%%%%%%%%%%%%%%%%%%%%%%%%%%%%%%%%%%%%%%%%%%%%%%%%%%%%%%%%%%%%%%%%%%%%%%%%%%%%
%%%%%%%%%%%%%%%%%%%%%%%%%%%%%%%%%%%%%%%%%%%%%%%%%%%%%%%%%%%%%%%%%%%%%%%%%%%%%%%%
\section{Delimitação do problema}%


Este trabalho encontra-se na área de Computação, com interdisciplinaridade na área de Educação, abordando o tema formação de professores de computação.

A grande informatização da sociedade atual requer a cada dia mais que as pessoas possuam melhor qualificação tecnológica para a vida social e profissional. A computação em si é uma área relativamente nova quando comparada às ciências clássicas, e a educação em computação é mais recente ainda, existindo ainda uma grande lacuna de pesquisa e investigação científica na área.

Entra então em debate a Licenciatura em Computação, que é um curso ainda com menor visibilidade e status em relação a outras denominações da Computação (ciência, engenharia, sistemas de informação, etc) por uma falta de explicitação de suas potencialidades. O mercado de trabalho do licenciado ainda está sendo estabelecido, não havendo uma disciplina de Computação na educação básica, o que restringe a criação de cargos e concursos públicos específicos para profissionais licenciados em Computação.

Entretanto, diante do contexto da atual sociedade da informação e do conhecimento, com grande necessidade de qualificação técnico-profissional em Computaçaõ, torna-se estratégico o estudo e a valorização do potencial de atuação do egresso de licenciatura em computação. É importante explicitar as várias formas de atuação já existentes para este profissional e ampliar os horizontes para as que ainda podem ser criadas. A grande necessidade social de qualificação técnica e profissional, crítica e reflexiva no âmbito tecnológico computacional, torna este um espaço rico para atuação e para investigação científica. Compreender a real intervenção social que o licenciado em computação pode realizar na sociedade.


Diante disso cabe a seguinte questão norteadora: qual a real necessidade social em investir na formação de professores de computação? Qual a real importância em formar licenciados em computação diante da atual sociedade da informação e do conhecimento?

Este trabalho não propõe a resolver os problemas na formação de professores de computação, mas sim em levantar o contexto social desta questão, explicitar problemas atuais e mostrar possíveis propostas, explicitando a necessidade de evolução da computação como um todo, da qualificação de pessoal e do desenvolvimento educacional, científico e tecnológico do país. 


%%%%%%%%%%%%%%%%%%%%%%%%%%%%%%%%%%%%%%%%%%%%%%%%%%%%%%%%%%%%%%%%%%%%%%%%%%%%%%%%
%%%%%%%%%%%%%%%%%%%%%%%%%%%%%%%%%%%%%%%%%%%%%%%%%%%%%%%%%%%%%%%%%%%%%%%%%%%%%%%%
%%%%%%%%%%%%%%%%%%%%%%%%%%%%%%%%%%%%%%%%%%%%%%%%%%%%%%%%%%%%%%%%%%%%%%%%%%%%%%%%
\section{Objetivo geral}%

Investigar a importância social da formação de professores de computação na sociedade atual, visando a valorização docente e o desenvolvimento tecnologógico e social do país.


%%%%%%%%%%%%%%%%%%%%%%%%%%%%%%%%%%%%%%%%%%%%%%%%%%%%%%%%%%%%%%%%%%%%%%%%%%%%%%%%
%%%%%%%%%%%%%%%%%%%%%%%%%%%%%%%%%%%%%%%%%%%%%%%%%%%%%%%%%%%%%%%%%%%%%%%%%%%%%%%%
%%%%%%%%%%%%%%%%%%%%%%%%%%%%%%%%%%%%%%%%%%%%%%%%%%%%%%%%%%%%%%%%%%%%%%%%%%%%%%%%
\section{Objetivos específicos}%

\begin{itemize}
	\item Contribuir com a valorização docente no país, especialmente do professor de computação, abordando a importância social de sua formação;
	\item Ampliar a visibilidade da importância dos investimentos no binômio Educação-Computação;
	\item Investigar e explicitar os possíveis espaços de atuação profissional dos egressos em Licenciatura em Computação enquanto profissionais da educação em computação, bem como fomentar a possibilidade de criação de novos espaços; 
	\item Divugar um curso jovem, relativamente desconhecido, mas que pode contribuir muito com a evolução da educação e da computação no país; e
	\item Pesquisar junto aos egressos da Licenciatura em Computação os benefícios que tal curso oferece à sociedade.
\end{itemize}


%%%%%%%%%%%%%%%%%%%%%%%%%%%%%%%%%%%%%%%%%%%%%%%%%%%%%%%%%%%%%%%%%%%%%%%%%%%%%%%%
%%%%%%%%%%%%%%%%%%%%%%%%%%%%%%%%%%%%%%%%%%%%%%%%%%%%%%%%%%%%%%%%%%%%%%%%%%%%%%%%
%%%%%%%%%%%%%%%%%%%%%%%%%%%%%%%%%%%%%%%%%%%%%%%%%%%%%%%%%%%%%%%%%%%%%%%%%%%%%%%%
\section{Metodologia}%

\begin{itemize}
	\item Pesquisa teórica e bibliográfica sobre conhecimentos científicos já produzidos na área de formação de professores de computação; e
	\item Pesquisa qualitativa com os egressos de Licenciatura em Computação da Universidade de Brasília nestes 20 anos de oferta do curso (1997 - 2017), visando esclarecer a importância social deste curso.
	
\end{itemize}

%%%%%%%%%%%%%%%%%%%%%%%%%%%%%%%%%%%%%%%%%%%%%%%%%%%%%%%%%%%%%%%%%%%%%%%%%%%%%%%%
%%%%%%%%%%%%%%%%%%%%%%%%%%%%%%%%%%%%%%%%%%%%%%%%%%%%%%%%%%%%%%%%%%%%%%%%%%%%%%%%
%%%%%%%%%%%%%%%%%%%%%%%%%%%%%%%%%%%%%%%%%%%%%%%%%%%%%%%%%%%%%%%%%%%%%%%%%%%%%%%%
\section{Resultados esperados}%


Compreender e explicitar a real identidade e potencialidade de atuação do profissional de Educação em Computação, visando uma maior valorização deste curso. Através dos egressos deste curso será possível analisar os benefícios sociais causados por ele, bem como a colaboração com a evolução da computação e da educação no país.



%%%%%%%%%%%%%%%%%%%%%%%%%%%%%%%%%%%%%%%%%%%%%%%%%%%%%%%%%%%%%%%%%%%%%%%%%%%%%%%%
%%%%%%%%%%%%%%%%%%%%%%%%%%%%%%%%%%%%%%%%%%%%%%%%%%%%%%%%%%%%%%%%%%%%%%%%%%%%%%%%
%%%%%%%%%%%%%%%%%%%%%%%%%%%%%%%%%%%%%%%%%%%%%%%%%%%%%%%%%%%%%%%%%%%%%%%%%%%%%%%%
\section{Organização do trabalho}%